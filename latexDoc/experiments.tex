% USAGE: copy contents of this file into experiments.tex


\section{Experiments and Results}

\subsection{Clustering}
We focused on clustering the top 2,000 most used hashtags. Decreasing this number would reduce the breadth of our clusters� coverage; on the other hand, increasing this number made the similiarity filtering step intractable. We performed several trials on the 5,000 most frequently used hashtags without running the step, but most of the hashtags ended up in one large cluster while the rest of the clusters consisted of single hashtags. We believe that this was caused by noisy connections, which inadvertently linked most of the hashtags closely together. This issue is removed during the filtering described earlier.

Only 1,557 hashtags remained after filtering. Unfortunately, there is no intuitive way to pick the correct number of clusters to generate. Spectral clustering does use $k$-means on the eigenvectors of the Laplacian matrix, which suggests that we could graph the within-cluster scatter and find the knee. Unfortunately, for spectral clustering, increasing $k$ increases the dimension of the eigenvectors that are evaluated in $k$-means. As a result, an increase in $k$ leads to an increase in within-cluster scatter. We tested several different values for $k$, and evaluated the size of the clusters that were formed. A small value for $k$ resulted in hashtag clusters that were not indicative of one single topic. Picking a large value of $k$ separated clusters that should actually be in the same group. We decided that K=300 provided the best trade-off between these two problems. 

In order to evaluate the clustering techniques we randomly selected 100 of the 300 clusters and score them manually on a scale of 1 to 5, with 1 representing a cluster that makes no sense and 5 being a perfect cluster. Factors that go into this rating include size of the cluster, number of matching hashtags that relate to the topics, and relation of topics if more than one topic arises. Some examples of clusters and their ratings are given below:

\begin{verbatim} 
bears beatles chargers cowboys fantasyfootball jets nyg packers panthers 
patriots raiders steelers vick vikings
Score: 5 (collection of all football teams)
amazing comic comics cool funny humor strange voss webcomic webcomics weird 
Score: 3 (clearly comics are represented, but there are random words such as 
amazing and weird)
crnc failedchildrensbooktitles fishy grizzly obamacare waterloo 
Score: 1 (list looks simplyrandom)
\end{verbatim}


\begin{table}[ht]
\caption{Ratings for the Clusters}
\centering
\begin{tabular}{c c c c c c c}
\hline\hline
Method & 1 & 2 & 3 & 4 & 5 & Average \\ [0.5ex] % inserts table %heading
\hline
Spectral & 21 & 6 & 10 & 21 & 35 & 3.23\\
\hline
Normalized Spectral & 36 & 7 & 8 & 15 & 34 & 3.04 \\
\hline
METIS & 30 & 13 & 13 & 13 & 15 & 2.22 \\[1ex]
\hline
\end{tabular}
\label{table:rating}
\end{table}


It is clear from Table 1 that spectral clusters performs the best, with normalized spectral close behind. Notably, normalized spectral tended to score around the two extremes, with not many scores in the middle values. METIS contained many clusters that should have been clustered together. For example, it contained five clusters of baseball teams, whereas the spectral methods grouped these teams into the same cluster. This implies that the clusters created by the spectral method would be more useful in identifying topics comprising may hashtags. It�s also possible that METIS would avoid this issue if $k$ was chosen to be a smaller number.

 

\subsection{Classification}
We split our data into 90$\%$ training data and 10$\%$ testing data. We do not perform cross-validation; however, anecdotally, results did not vary by more than a couple of percentage point over several different divisions of the data.

We use majority-vote classification accuracy as a baseline to measure our performance against. The majority-vote accuracy is simply the percentage of the data points whose cluster is the most common cluster among the data points.

\begin{table}[ht]
\caption{Classification Performance}
\centering
\begin{tabular}{c c c c c c}
\hline\hline
& Bern. N.B.
& LDA
& SVM
& Majority
& SVM, no PCA \\ [0.5ex]
\hline
Spectral 
& 26.5 & 18.3 & 25.1 & 11.7 &11.8 \\
\hline
Spectral w/ Cos. Sim.
& 26.5 & 18.6 & 24.1 & 11.1 & 10.8 \\
\hline
Norm. Spectral
& 21.4 & 17.2 & 25.2 & 17.9 & 17.9 \\
\hline
Norm. Spectral w/ Cos. Sim.
& 22.4 & 17.3 & 25.1 & 17.4 & 17.6 \\
\hline
METIS
& 14.8 & 11.8 & 13.8 & 4.1 & 4.1 \\
\hline
METIS w/ Cos. Sim.
& 14.9 & 12.1 & 14.1 & 3.9 & 3.9 \\
\hline
\end{tabular}
\label{table:rating}
\end{table}

\subsection{Discussion}

Over all clustering techniques, all classification techniques using PCA-reduced data perform better than majority vote; SVM without PCA reduction performs no better than majority vote. This shows that dimensionality reduction serves two purposes: it makes the problem computationally feasible, and it improves performance by capturing regularities in the data.

Our data show that the classification techniques performed similarly well when given clusters that included hashtags added via cosine-similarity measures as when they did not. This suggests that the cosine similarity technique allows us to expand the breadth of clustering without negatively affecting the semantic cohesiveness of the clusters.

Although all three classification techniques over PCA-reduced data perform better than majority vote, the Bernoulli Naive Bayes and SVM classifiers perform better than LDA over all clustering techniques. This suggests that when PCA reduction is applied the data points are not easily linearly separable, but that limitation may be overcome in the case of Bernoulli Naive Bayes due to the binarization of the data, or in the case of the SVM classifier by using a non-linear kernel. 

Under all classification techniques, data clustered by the METIS algorithm had a lower classification success rate than all other clustering algorithms. In our manual examination of the clusters produced by each technique, we saw that METIS produces clusters that seem less semantically cohesive than the other techniques. We conjecture that classification accuracy is therefore an indicator of the quality of a clustering technique. Alternatively, METIS's worse performance may be a result of the distribution of tags to clusters; the lower accuracy of the majority vote classifier indicates that hashtags are more evenly spread among clusters under METIS than under the other techniques, which may result in more difficulty performing classification regardless of the semantic cohesiveness of the clusters.